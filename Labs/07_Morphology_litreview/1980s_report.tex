\documentclass[11pt,a4paper]{article}

%-----------------------------------------------------------
% Encoding and Fonts
\usepackage[utf8]{inputenc}           % UTF-8 encoding
\usepackage[T1]{fontenc}              % Font encoding
%\usepackage{lmodern}
\usepackage{newtxtext,newtxmath}      % Times font + math

%-----------------------------------------------------------
% Bibliography
\usepackage[style=apa, backend=biber]{biblatex}
\addbibresource{references.biblatex} % BibLaTeX bibliography file

%-----------------------------------------------------------
% Graphics and Tables
\usepackage{graphicx}                  % Include graphics
\usepackage{booktabs}                  % Nice tables
\usepackage{caption}                   % Captions customization

%-----------------------------------------------------------
% Page Layout
\usepackage[margin=1in]{geometry}      % Page margins
\setlength{\parskip}{0.3em}            % Space between paragraphs
\setlength{\parindent}{0pt}            % No paragraph indentation

%-----------------------------------------------------------
% Line spacing
\usepackage{setspace}
\setstretch{1.0}                       % Custom line spacing

%-----------------------------------------------------------
% Colors, hyperlinks, and math
\usepackage{xcolor} 
\usepackage[colorlinks=true, citecolor=black, linkcolor=blue, urlcolor=blue]{hyperref} 
\usepackage{amsmath}

%-----------------------------------------------------------
% Header and Footer
\usepackage{fancyhdr}
\pagestyle{fancy}
\fancyhf{} % Clear default header/footer
\fancyhead[L]{\textcolor{black!60}{Ethel Ogallo}} 
\fancyhead[R]{\textcolor{black!60}{ERSIP: Mathematical Morphology}} 
\fancyfoot[C]{\thepage} 
%\renewcommand{\headrulewidth}{0.pt} % Remove header line

%-----------------------------------------------------------
\title{\LARGE \textbf{Mathematical Morphology in Remote Sensing: 1980s-2000s}}
\author{{Ethel Ogallo}}
\date{{\today}}

\begin{document}

\maketitle

%\section*{Introduction}

Mathematical Morphology (MM), founded by Matheron and Serra in the 1967, is a non-linear image analysis framework based on set theory and  
lattice structures. This report looks at the promises and limitations of MM as described in research from the 1980s and the 2000s, the challenges, evolution and 
advancements in MM methods into the 2000s, and their limitations given the current state of the field.

%-----------------------------------------------------------
\section*{Mathematical Morphology in the 1980s}

\subsection*{Promises of MM in the 1980s}

In the 1980s , classical image processing methods based on linear transformations were the norm however they were limited to 
pixel-level operations and such a need apply automatic detection of shapes and structures arose. This is where Mathematical 
Morphology (MM) comes into play because of its ability to process geometric structures in images using set operators acting on the
objects independent of shape and grey level. 

The work of \parencite{destivalMathematicalMorphology1986} explored MM effectiveness in extracting linear features, such as roads and ridges, 
highlighting that it was best used in combination with classical methods within a single pipeline which \parencite{flouzatReviewImageAnalysis1989} 
further emphasized on by demonstrating that it offered sequential operations allowing for more flexible and efficient processing pipelines. their
studies also demonstrated the promise of MM in automating feature extraction with the use of structuring elements to replace manual 
photointerpratation was was very time consuming. 

In addition to linear feature extraction, MM was promising in local contrast detection and boundary delineation which was capable because of 
morphological gradients, the difference between dilation and erosion highlighting the changes in intensity which is useful
for edge detection. 

Another promise was on the ability of MM to handle image noise and texture by using morphological filters such as opening and closings. 
These transformations removed small scale variations while still preserving larger important structures in the image. This feature of MM 
was very useful for speckle removal in radar imagery and spectral analysis. 

Multi-scale image processing was another area highlighted by \parencite{flouzatReviewImageAnalysis1989} demonstrating MM's versatility in
adapting to process different level of image represenattion i.e. low-level(pixel-level) to high level (object-level) analysis. 


\subsection*{Limitations given current knowledge}

Although promising, early MM approaches in the 1980s had several limitations when viewed from today's knowledge and applications point of view.

MM operations were very dependent on the choice, size and order of structuring elements which had to be manually defined . 
This dependency reduced automation and impacted results which makes the process less reproducible. 

Early MM methods were developed for greyscale and binary imagery and thus would struggle with multispectral or hyperspectral data which is The
standard today. Additionally, feature extraction also relied on simple semantic assumptions, such as treating roads as bright lines or 
ridges as highly textured areas, which proved to be ineffective to complex landscapes like urban areas. 

MM methods in this era were limited in adaptability and statistical learning. This 
meant that noisy or mixed pixels were not handled properly compared to modern machine learning and deep learning methods which can learn from
the data itself.

While decomposition into simpler operations improved computational efficiency at the time, early MM approaches are 
inefficient for larger or very high-resolution images, as the basic morphological filters proposed were insufficient for capturing 
complex object boundaries, limiting their efficiency in heterogeneous landscapes.

%-----------------------------------------------------------
\section*{Mathematical Morphology in the 2000s}

\subsection*{Evolution of MM from the 1980s}

The 2000s brought alot of advancement on Mathematical Morphology (MM) improving on the methods defined in the 1980s as well as 
addressing the limitations of those methods.

A major improvement was the use of connected operators which use morphological characteristics over simple structuring elements 
(SE)-based operators because they preserve object shapes better by focusing on spatial relationships and their connectivity. \parencite{soilleAdvancesMathematicalMorphology2002}. 

The research in the 1980s also highlighted that the results were sensitive to noise and texture variations. To address the issue 
of noise sensitivity the 2000s introduced the use of geodesic metrics over euclidean metrics to constrain operations within a 
mask or object thus following its true shape and hence reducing sensitivity to noise \parencite{soilleAdvancesMathematicalMorphology2002}. 
Additionally, the extension of the multiscale framework approach proposed by \parencite{pesaresiNewApproachMorphological2001} which uses leveling, 
morphological profiles and their derivatives to extract features without relying on gradient computation mitigates the noise sensitivity issue. 

Remote sensing technology advanced in the 2000s offering color imagery which required MM to be extended beyond the early operations done on 
grayscale and binary imagery. One development was in the extension of MM to multivariate MM which allowed operations on multichannel images
using different approaches like rank-based morphological filters which allows the user to tune the flexibility of the structuring element 
fit by specifying how many pixels are allowed not to fit the image structures, and order-independent skeletonization which produced consistent
results regardless of the order of pixel processing as was a main limitation of early MM methods \parencite{soilleAdvancesMathematicalMorphology2002}.

Several strategies of multivariate morphology were also introduced as a solution for the limitations of early MM methods developed for single-band images.
These included approaches like vector ordering, fuzzy morphology and graph-based methods for efficient feature extraction and classification specifically 
for multichannel imagery \parencite{aptoulaComparativeStudyMultivariate2007}.


\subsection*{Challenges in the MM community in the 2000s}

The Mathematical Morphological (MM) community faced a couple of challneges in the 2000s. These challenges were largely due to the development in 
the remote sensing technology which brings an increase in imagery data in terms of data types and data volume.

One challenge was in the extension of MM from grayscale imagery whose pixel values could be easily ordered using
simple operations like min and max to multivalued imagery where this ordering is not as simple because here the pixels are 
 represented as vectors which are not inherently ordered. Several strategies were developed to mitigate this including total, marginal or partial orderings but there still was
not a standard solution that worked for all multivalued data types or applications \parencite{aptoulaComparativeStudyMultivariate2007, soilleAdvancesMathematicalMorphology2002} 
limimting generalization to complex datasets.

The challenge of over-segmentation in complex scenes such as urban areas carried over from 1980s where MM methods like watershed segmentation and 
edge detection were prone to this due to texture effects leading to fragmented results. Methods such as multiscale analysis and marker-controlled watershed
segmentation were proposed to guide segmentation process \parencite{pesaresiNewApproachMorphological2001, soilleAdvancesMathematicalMorphology2002} however the 
a new challenge of parameter sensitivity was created.

Computational efficiency persisted as a challenge especially as image quality increased and datasets grew larger. Proposed MM methods like 
derivatives of morpholical profiles and directional structuring elements required multiple passes over the data which increased computational
costs \parencite{soilleAdvancesMathematicalMorphology2002}.


\subsection*{Promises of MM in the 2000s}

The promises of Mathematical Morphology (MM) in the 2000s were centered around working with multichannel images.A 
major development was the use of morphological characteristics of connected operators for segmentation which preserved object shapes better.
and the use of residuals from morphological transformations computed on geodesic metrics which enabled better segmentation
of high resolution imagery.

Another promise was using morphological profiles and their derivatives \parencite{pesaresiNewApproachMorphological2001} which provided 
scale-dependent features without having to rely on gradient computation which was unsatble. This proved to be
effective in reconstruction-based operations like classification and segmentation because it offered more accurate shape descriptions
and reduced over segmentation.

The extension of MM to multivariate MM introduced vector-ordering schemes that allowed operators to work directly on multivalued data \parencite{aptoulaComparativeStudyMultivariate2007}. 
These schemes included total ordering, marginal ordering, and graph-based methods which enabled MM to handle multichannel data.
Additionally, the directional and multiscale operators enabled analysis of spatial structures at varying scales and orientations useful for
detecting anisotopic structures and elongated features like leinear segments \parencite{soilleAdvancesMathematicalMorphology2002}. 
 
The 2000s promise of consistent results especially in complex scenes was achieved by using order-independent skeletonization which reduced sensitivity
to pixel processing order while self-dual and self-complementary morphological filters provided robustness to local contrast variations which ensured
consistent results\parencite{soilleAdvancesMathematicalMorphology2002}.

The integration of MM with other techniques such as machine learning and GIS also promised a more adaptive image processing \parencite{soilleAdvancesMathematicalMorphology2002}. 
By combining the pros of MM in shape and structure analysis with the learning capabilities of machine learning, an improved classification and 
feature extraction systems could be developed. 

\subsection*{Limitations given current knowledge}

From the current point of view, the Mathematical Morphology (MM) approaches developed in the 2000s have some limitations. The issue of 
computational cost when it pertains to the methods persisted. Given the current state of technology where we have hyperspectral data as well 
as an enormous amount of data, the computational intensive nature of these methods would only be increased further especially 
for real time or large scale remote sensing workflows.

MM methods that were parameter-free or data-driven was still largely undeveloped as many of the methods developed in the 2000s required proper selection of parameters such as 
the size, shape and orientation of structuring elements which played a key role in the outcome especially in multi-scale segmentation.

Another limimtation is on vector ordering where there was no universally applicable method in multivariate morphology. 
This is because the stragtegies proposed produced different results given their specific assumptions for example lexicographical 
ordering had inconsistencies based on the order of channels.

%\section*{References}
\printbibliography

\end{document}