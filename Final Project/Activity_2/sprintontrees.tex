\documentclass[11pt,a4paper]{article}

%-----------------------------------------------------------
% Encoding and Fonts
\usepackage[utf8]{inputenc}
\usepackage[T1]{fontenc}
\usepackage{newtxtext,newtxmath}

%-----------------------------------------------------------
% Bibliography
\usepackage[style=apa, backend=biber]{biblatex}
\addbibresource{references.biblatex} % BibLaTeX bibliography file

%-----------------------------------------------------------
% Graphics and Tables
\usepackage{graphicx}
\usepackage{booktabs}
\usepackage{caption}

%-----------------------------------------------------------
% Page Layout
\usepackage[margin=1in]{geometry}
\setlength{\parskip}{0.3em}
\setlength{\parindent}{0pt}

%-----------------------------------------------------------
% Line spacing
\usepackage{setspace}
\setstretch{1.0}

%-----------------------------------------------------------
% Colors, hyperlinks, and math
\usepackage{xcolor}
\usepackage[colorlinks=true, citecolor=black, linkcolor=blue, urlcolor=blue]{hyperref}
\usepackage{amsmath}


%-----------------------------------------------------------
\begin{document}

% Manual title (normal document style)
% Name, date, and tagline (left-aligned, indented)
Ethel Ogallo \par
\today \par
\textit{Efficient Remote Sensing Image Processing} \par

\vspace{1.2em}

% Main title (centered)
\begin{center}
{\Large\bfseries Tree-based Image Processing: Pros and Cons }
\end{center}


% \section*{Overview}
Tree based image processing involves analyzing image via a hierarchy of connected components represented as trees. 
These heirachies can be either inclusion trees such as max and min trees or partition trees such as binary partition trees 
and $\alpha$-trees. This framework allows for efficient representation and analysis of images at multiple scales and resolutions includes 
tools such as attribute profiles which are a generalization of morphological profiles by replacing structuring 
elements with attribute filters operating on connected components and pattern spectra which provides the statistical 
distribution of the attribute values. 


\section*{Advantages}
Some of the advantages of tree based image processing include:

Computational efficiency and scalability beacuse it requires fewer operations. This is possible because the tree structure is 
constructed only once and further analysis can be performed on the tree rather than the entire image since the 
trees store the hierachical relationships between regions in the image. The tree construction algorithms achieve a $O(N)$ or $O(N \log N)$ 
complexity which is an improvement from the morphological profiles which require $O(N²)$ complexity, 
where N is the number of pixels in the image

Flexibility in extraction of features because of the use of computable attributes such as geometric (e.g. area, volume) and statistical 
(e.g. standard deviation, variance) of the connected components. This allows for different types of features to be extracted given an application/task.

Self-duality where tree structures such as Tree-of-Shapes($ToS$) of $\alpha$-tree represent both bright and dark
structures simulatneosuly i.e. the local extrema are processed together and represented in a single tree. This leads to a 
complete reconstruction of the image without redudancy and loss of information.

In image classification applications, it has proved to be more accurate because the use of attribute
profiles which can capture the complex spatial relationships and structures in images more effectively. They have also 
been integrated into machine learning and deep learning as they are used as features in models reducing the need for large training datasets. 

\section*{Limitations}
Some of the limitations of tree based image processing include:

Sensitivity to parameters such as choice of attributes and threshold for filtering which can impact the result of the analysis.
Careful selection of these parameters is necesssary depending on the data. Manual thresholds may not generalize well
across different datasets whereas automatic methods may add onto the overall compuational cost.

Chaining effect from trees like the $\alpha$-tree where connected components may merge dissimilar regions due to its nature which 
can impact the accuracy of the analysis. On the other hand, oversegmentation can occur when the tree structure, i.e binary partition trees, 
creates too many small regions that do not correspond to meaningful objects in the image.

Increase in memory usage and redundancy due to generation of attribute profiles especially when multiple attributes and thresholds are combined. 
Pattern spectra can remedy this but it can lose essential spatial arrangement which may be necessary for some tasks.
 

% \section*{References}
% \parencite{dallamuraMorphologicalAttributeProfiles2010}.
% \parencite{bosiljPartitionInclusionHierarchies2018}.
% \parencite{maiaClassificationRemoteSensing2021}.
% \printbibliography

\end{document}
